In this section we will cover the compilation part of open\+Zia. We will see how to compile it independently, how to add it to your actual C\+Make project and how to automate module compilation.

\subsection*{Independent compilation}

You may first want to assert that open\+Zia compiles well in your system. The first dependency is to check your C++ compiler, it must be able to support {\bfseries{all C++17 features}}. You must also have {\bfseries{C\+Make}} installed with a lower build system like {\bfseries{Makefiles}} or {\bfseries{Visual studio solutions}}. If you wish to start unit tests, you must install {\bfseries{Criterion}} unit-\/testing library. For the coverage, you will need {\bfseries{gcovr}}.

First, you need to clone the repository and create a build directory\+: 
\begin{DoxyCode}{0}
\DoxyCodeLine{git clone git https://github.com/MatthieuMv/openZia.git}
\DoxyCodeLine{cd openZia \&\& mkdir Build \&\& cd Build}
\end{DoxyCode}


Then, for a simple library build use the following commands 
\begin{DoxyCode}{0}
\DoxyCodeLine{cmake ..}
\DoxyCodeLine{cmake --build .}
\end{DoxyCode}
 If you want to run test or see coverage please watch below.

\subsection*{Add open\+Zia to your C\+Make project}

First you need add open\+Zia\textquotesingle{}s repository to your submodules. 
\begin{DoxyCode}{0}
\DoxyCodeLine{git submodule add https://github.com/MatthieuMv/openZia.git}
\end{DoxyCode}


Next you need to {\bfseries{automate compilation}} of open\+Zia in your {\bfseries{C\+Make\+Lists.\+txt}}, and link your program to it. 
\begin{DoxyCode}{0}
\DoxyCodeLine{\# CMakeLists.txt}
\DoxyCodeLine{cmake\_minimum\_required(VERSION 3.0)}
\DoxyCodeLine{project(Server)}
\DoxyCodeLine{}
\DoxyCodeLine{\# Include directly the openZia.cmake to your CMake build}
\DoxyCodeLine{\# This include gives 'openZiaLibs' and 'openZiaIncludes' variables.}
\DoxyCodeLine{include(openZia/openZia/openZia.cmake)}
\DoxyCodeLine{}
\DoxyCodeLine{\# Create the executable and link openZia library to it}
\DoxyCodeLine{add\_executable(Server main.cpp)}
\DoxyCodeLine{target\_link\_libraries(Server \$\{openZiaLibs\})}
\DoxyCodeLine{target\_include\_directories(Server PRIVATE \$\{openZiaIncludes\}))}
\end{DoxyCode}


\subsection*{Automate module compilation}

A good practice is to create a directory for each module, and include their $\ast$$\ast$.cmake$\ast$$\ast$ file individually. I assume you already included {\bfseries{open\+Zia.\+cmake}} in your {\bfseries{C\+Make\+Lists.\+txt}} and that you have created a directory named {\bfseries{Template}} which contains the following {\bfseries{Template.\+cmake}}, {\bfseries{Template.\+hpp}} and {\bfseries{Template.\+cpp}}. 
\begin{DoxyCode}{0}
\DoxyCodeLine{\# Template.cmake}
\DoxyCodeLine{project(Template)}
\DoxyCodeLine{}
\DoxyCodeLine{\# Save this directory as TemplateSourcesDir}
\DoxyCodeLine{get\_filename\_component(TemplateSourcesDir \$\{CMAKE\_CURRENT\_LIST\_FILE\}  PATH)}
\DoxyCodeLine{}
\DoxyCodeLine{set(TemplateSources}
\DoxyCodeLine{    \$\{TemplateSourcesDir\}/Template.hpp}
\DoxyCodeLine{    \$\{TemplateSourcesDir\}/Template.cpp}
\DoxyCodeLine{)}
\DoxyCodeLine{}
\DoxyCodeLine{\# Compile the module, and link it to openZia}
\DoxyCodeLine{add\_library(\$\{PROJECT\_NAME\} SHARED \$\{TemplateSources\})}
\DoxyCodeLine{target\_link\_libraries(\$\{PROJECT\_NAME\} \$\{openZiaLibs\})}
\DoxyCodeLine{target\_include\_directories(\$\{PROJECT\_NAME\} PRIVATE \$\{openZiaIncludes\})}
\DoxyCodeLine{}
\DoxyCodeLine{\# Copy the library into the binary modules directory}
\DoxyCodeLine{set\_target\_properties(\$\{PROJECT\_NAME\} PROPERTIES}
\DoxyCodeLine{    LIBRARY\_OUTPUT\_DIRECTORY  \$\{CMAKE\_BINARY\_DIR\}/Modules\}}
\DoxyCodeLine{)}
\end{DoxyCode}


\subsection*{Windows and the winsock version}

Even if the A\+PI uses windows only for dynamic load, you may need the following configuration \+: 
\begin{DoxyCode}{0}
\DoxyCodeLine{ \{C++\}}
\DoxyCodeLine{// Windows socket version}
\DoxyCodeLine{\#include <winsock2.h> // or <winsock.h>}
\DoxyCodeLine{// Include windows header after selecting a winsock version !}
\DoxyCodeLine{\#include <Windows.h>}
\end{DoxyCode}
 In that case, you can define a macro named {\bfseries{O\+P\+E\+N\+Z\+I\+A\+\_\+\+P\+R\+E\+\_\+\+W\+I\+N\+D\+O\+W\+S\+\_\+\+I\+N\+C\+L\+U\+DE}} to your build system to import a custom header.

For example you can add the following line into your cmake module. 
\begin{DoxyCode}{0}
\DoxyCodeLine{\# Please note that quotes matters /!\(\backslash\)}
\DoxyCodeLine{target\_compile\_definitions(\$\{PROJECT\_NAME\} PUBLIC OPENZIA\_PRE\_WINDOWS\_INCLUDE="winsock2.h")}
\end{DoxyCode}


\subsection*{Unit tests and code coverage}

If you want to start unit tests replace independent build instruction by\+: 
\begin{DoxyCode}{0}
\DoxyCodeLine{cmake .. -DOPENZIA\_TESTS=TRUE}
\DoxyCodeLine{cmake --build .}
\DoxyCodeLine{./openZiaTests}
\end{DoxyCode}


If you want code coverage then\+: 
\begin{DoxyCode}{0}
\DoxyCodeLine{cmake .. -DOPENZIA\_TESTS=TRUE -DOPENZIA\_COVERAGE=TRUE}
\DoxyCodeLine{cmake --build .}
\DoxyCodeLine{./openZiaTests}
\DoxyCodeLine{cd ..}
\DoxyCodeLine{gcovr --exclude Tests}
\end{DoxyCode}
 